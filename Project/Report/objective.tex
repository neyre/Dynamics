\section{Learning Objectives}
In our double pendulum simulation, we found that our model predicted the behavior of the experimental setup quite well for the first few seconds, but soon after diverged rapidly into a completely different path. We pointed to the lack of friction in our model as the likely cause of this error. Even though friction has a small effect, the chaotic nature of the double pendulum means that any small perturbation can vastly change the dynamics.

We wanted to further investigate this outstanding question for our final project. As we had previously analyzed a double pendulum using Newtonian methods, we chose to model a triple pendulum with energy methods in order to cover new material. While energy methods are useful only in a small subset of real-world dynamics problems, they are nonetheless an interesting and wildly different approach. Furthermore, we were interested in learning how to represent friction in a complex system, which we have done little in the class so far. 

Overall, for our project we are interested in studying how changes in damping parameters of a system change the dynamics. This will allow us to investigate our hypothesis that the chaotic behaviour of a muliple pendulum system causes wildly divergent behaviour with the introdcution of small friction forces.  Furthermore, we want to learn more about the use of energy methods with nonconservative systems.
