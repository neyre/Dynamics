\section{Reflection}
Leaning about Lagrangian mechanics from an operational perspective was interesting. In some ways it simplifies the derivation of the equations of motion: we had to solve three equations with Lagrangian mechanics instead of the nine we would have needed to do it with Newtonian mechanics. For mechanical systems you are actually trying to design, the Newtonian perspective is vastly better because it gives insight into internal forces which might require structural changes and which variables are most important to the system. We learned that while the Lagrangian method can simplify calculating the dynamics of a system significantly, it is really only useful for a small subset of highly theoretical problems.

Even very small drag forcesmchanged the dynamics of the triple pendulum significantly. This drives home the idea that a triple pendulum is a chaotic system. If small changes in drag create wildly different behaviours you need different techniques to learn someting from your model, because it is unlikely that it will give a highly accurate prediction of the actual behaviour.

We also learned about using Mathematica to do large amounts of simple algebra and calculus. The ability to define intermediate functions allowed us to work on a higher level of abstraction than the gory algebraic details. For example, this allowed us to simple write $v_1$ instead of the long expression corresponding to this in terms of base quantities. This allowed us to see the key components of what we were doing without having to slog through the algebraic detail. However, it was absolutely neccesary to have a strong understanding of what we wanted to do before using Mathematica. This allowed us to better debug the code and interpret its output.
