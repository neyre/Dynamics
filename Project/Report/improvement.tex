\section{Improvement}
The main improvements which could be made on the model are for more accurate representations of the damping forces on the system.  This current model takes into effect viscous damping.  A future model could extend this by also including aerodynamic drag as the bars swing through the air.

Furthermore, when testing the system we found that when the pendulum encountered rapid, chaotic movements, the bars of the pendulum encountered dramatic out of plane vibrations.  These vibrations transferred into the table onto which the system was mounted, shaking the table and removing energy from the system.  Although this is a rather complicated damping effect, large quantities of energy were likely lost due to these vibrations and a model of the experimental setup would not be complete without including them.

In addition to adding better models of damping, the model could be made more general by taking it into the third dimension with a three-dimensional triple pendulum.

Finally, there are a number of small physical effects which could be added to the model.  For example, the model could be extended to include the elongation of the steel pendulum rods, the change in gravitational constant as the rods change in elevation or even the gravitational effects of the bars on each other.  However, most of these effects are negligible and would not significantly change the results.
