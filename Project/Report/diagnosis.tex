\section{Diagnosis}
The largest problems we encountered with this project involved our infamiliarity with Mathematica, the symbolic algebra system we used to solve for the equations of motion. When we started in this project, we knew just enough to get us in trouble. Once we had the Lagrangian, we had difficulty in applying Lagrange's equation to it, as it requires taking derivatives of a function with respect to another function. This combined with our general unfamiliarity with Mathematica syntax made it difficult for us to fix problems. To move forward, we had to step back and go through a few basic Mathematica tutorials. Once we had a stronger grasp on the fundamentals, we found it much easier to solve the problems we were encountering, even though they existed on a higher level of abstraction.

Once we had the equations of motion, we had difficulty in calculating the energies of the system. We had solved the dynamics both using lagrangians and vectors, and our notation did not match between them. Small ambiguities such as the sign of gravity and the definition of our axes were not clearly defined when we began the derivations. This made interpreting the results of the simulation difficult. When we went back to the beginning and explicitly wrote out our system definitions, we were able to more clearly interpret and use our results.

The motion capture system we used to capture experimental data of a real triple pendulum did not tolerate high angular velocities well. It would lose track of links when they began moving too quickly. We could not find an effective fix for this. We made due with what we had by only testing initial conditions which did not produce overly large velocities in the system.
