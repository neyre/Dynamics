\section{Future Usage}
The step between a double and a triple pendulum is a rather minor one. Mostly it involves more of the same type of mathematics. At most this is character building if solving by hand, but because we used a computer algebra system, it made little difference. It is probably better to instead study a double pendulum. That said, after having done a bob double pendulum, a compound double pendulum, and a triple pendulum, it is interesting to see that the process is really exactly the same. 

However, the introduction of drag into the model was quite interesting. Seeing how this changed the dynamics of the system drove home the chaotic behaviour of the pendulum. Furthermore, thinking about how to represent a drag torque in the mathematical model of the system gave us a greater understanding of what torque is, although this did not make it into the final report.
An appropriate problem state for our project might be:

Consider a planar triple pendulum which consists of three massed bars joined by pivots. The pendulum is fixed to a stable structure at the top bar. Each section has mass $m_1$, $m_2$, and $m_3$ with moments of inertia about the center of mass $I_1$ $I_2$ and $I_3$. 

\begin{itemize}
\item Write the Lagrangian of the system in terms of the angular positions and velocities of the bars.
\item Determine the angular acceleration of the bars with Lagrange's equation.
\item Write a MATLAB script to simulate the motion of the pendulum. Validate your results with a number of initial conditions and show that energy is conserved.
\item Now include damping using Rayleigh's Dissipation Function or some other method. What form of damping would be most accurate? Viscous or aerodyanmic, or something else? What effect does the inclusion of drag have on the dynamics of the system?
\end{itemize}
